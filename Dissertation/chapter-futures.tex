\chapter{Conclusions}
\label{chp:conclusions}
A fast and accurate model for energy consumption and thermal envelope in
a server is critical to understanding and solving the power management
challenges unique in dense servers.  In this paper, we have introduced a
comprehensive model of energy consumption by servers as a continuous
system of differential equations.  The model measures energy input to
the system as a function of the work done for completing tasks being
gauged and the residual thermal energy given off by the system as a
result.  Traffic on the system bus, misses in the L2 cache, CPU
temperatures, and ambient temperatures are combined together to create a
model, which can be employed to manage the processor thermal envelope.

The model serves as a predictive tool by approximating observed
performance metrics in a discrete time series for estimating future
metrics, and thus corresponding energy consumption amounts.  It was
found through experimental validation that commonly used techniques of
regressive time series forecasting, while attractive because of their
simplicity, inadequately capture the non-linear and chaotic dynamics of
metric readings for typical server systems.  Therefore, a chaotic time
series approximation for run-time power consumption is adopted to arrive
at Chaotic Attractor Prediction (CAP), which exhibits polynomial time
complexity.  Our proposed model is the first step towards building
solutions for power and thermal management in data centers usually
housing many servers.
\section{Future Directions}
\label{sec:future-directions}
The topics of power and thermal management in large scale computing have
only recently begun to receive systematic attention.  As such, we
anticipate that the work in this Prospectus will serve as a starting
point for a longer-term research program.   Examples of longer-term
questions that will need to be addressed include:
\begin{itemize}
\item Our Thermal-Aware Scheduler considers scheduling only within a single
  server blade. High-performance computing applications distribute
  workload across multiple environments using interfaces such as MPI and
  OpenMP.  A topic for further research is how to extend the TCAP and
  Thermal-Aware Scheduling into such environments.
\item A related topic for the future involves the effect of
  operating-system virtualization on Thermal-Aware Scheduling.  Some
  attention has been applied to this area in recent work
  \cite{Merkel2010} but questions remain open considering issues of
  scheduler interference between host and virtual operating systems and
  the impact of virtualization on power management in the clustered,
  grid, and cloud environment.
\item An interesting extension of the system model described in
  \ref{chp:modelstruct} involves considering whether we can quantify the
  energy expended per bit transfer as data moves through the
  system. From this, we can build a more detailed analytical model of
  the thermodynamics of the system than the current practice of
  RC-thermal modeling \cite{Skadron2004}.  
\end{itemize}

% Following comment block used by GNU-EMACS and AUCTEX packages
% Please do not remove.
%%% Local Variables: 
%%% mode: latex
%%% TeX-master: "dissertation.tex"
%%% TeX-PDF-mode: t
%%% TeX-source-correlate-mode: t
%%% End: 
