%
% File:     chapter-appendix1.tex
% Author:   awl8049
% Revision: $$
%
This appendix describes three regression-based prediction models: a linear
AR(1) model \cite{Box1994}, a MARS model \cite{Friedman1991}, and an
EWMA model. 
Each model is an  approximations to the dynamic system in
\equationname~(\ref{eq:linmodel}), following regressive combinations of
five energy contributors to a server, as given in 
\equationname~(\ref{eq:tseries})~\cite{Lewis2008}.

The same data sets used to generate the chaotic model were used to
create the AR(1) model. Two methods were considered for consolidation:
arithmetic mean (average) and geometric mean.  Trial models were
constructed using each method and a statistical analysis of variance was
performed to determine which model generated the best fit to the
collected data with a time interval $t=5$ seconds.  Note that the statistical coefficients need to be computed
only once using some benchmarks, for a given server architecture.
The coefficients obtained can then be provided through either
the system firmware or the operating system kernel for use in the server
executing any application.

Under linear auto-regression, energy consumed by the processor for the
AMD server, $E_{proc}^{AMD}$ as defined by
\equationname~(\ref{eq:procpwr2}), is a \textit{linear combination} of
$MP_{proc}^{AMD}$ measures (stated in Section~\ref{sec:variables}) as \cite{Lewis2008}:
\begin{equation*}
  \label{eq:apxproc}
  E_{proc}^{AMD} \approx 0.49*T_{C_{0}}+0.50*T_{C_{1}}+0.01*HT_{1}. 
\end{equation*}

For the Intel server, its energy consumption by the processor,
$E_{proc}^{Intel}$, is a function of $MP_{proc}^{Intel}$ (detailed in
Section~\ref{sec:variables}), leading to its estimated energy as follows:
\begin{equation*}
  \label{eq:apxpr}
  E_{proc}^{Intel} \approx 2.29*T_{C_{0}}+0.03*T_{C_{1}}+0.52*QPL_{C}.
\end{equation*}
 
In a similar fashion, energy consumed by the memory subsystem in the AMD server,
$E_{mem}^{AMD}$, is a function of $MP_{mem}^{AMD}$, yielding
\begin{equation*}
  \label{eq:apxmem}
  E_{mem}^{AMD} \approx 0.01*HT_{2}+0.003*CM_{0}+0.003*CM_{1}+0.014*CM_{2}+0.01*CM_{3}.
\end{equation*}

Energy consumption for the memory subsystem in the Intel server,
$E_{mem}^{Intel}$, is a function of $MP_{mem}^{Intel}$, giving rise to
\begin{equation*}
  E_{mem}^{Intel} \approx 0.52*QPL_{IO}+0.35*CM_{0}+0.31*CM_{1}. 
\end{equation*}

Energy consumed as a result of disk activities in the AMD server (or the
Intel server) is a function of $MP_{hdd}^{AMD}$ (or $MP_{hdd}^{Intel}$),
arriving at 
\begin{equation*}
E_{hdd}^{AMD} \approx 0.014*D_{r}+0.007*D_{w}
\end{equation*}
and
\begin{equation*}
E_{hdd}^{Intel} \approx 0.01*D_{r}+0.01*D_{w}.
\end{equation*}

Energy consumed by the board in the AMD server is a function of $MP_{board}^{AMD}$,
whose components are added in a linearly weighted fashion to derive
$E_{board}^{AMD}$ (or $E_{board}^{Intel}$), as follows:
\begin{equation*}
E_{board}^{AMD} \approx 0.101+0.81*T_{A_{0}}+0.62*T_{A_{1}}
\end{equation*}
and
\begin{equation*}
E_{board}^{Intel} \approx 2.53+0.03*T_{A_{0}}+0.01*T_{A_{1}}+0.01*T_{A_{2}}.
\end{equation*}

Finally, energy consumed by electromechanical elements in the AMD server, $E_{em}^{AMD}$,
is a linear function of $MP_{em}^{AMD}$, leading to
\begin{equation*}
E_{em}^{AMD} \approx 0.001*F_{C}+0.001*F_{M}.
\end{equation*}
Similarly, energy consumption attributed to electromechanical elements in the Intel server, $E_{em}^{Intel}$, equals
\begin{align*}
E_{em}^{Intel}\approx4.85*F_{C}&+6.61*F_{M2a}+3.92*F_{M2b}+0.28*F_{M3a}+0.52*F_{M3b}\\
            &+0.01*F_{M4a}+0.01*F_{M4b}+0.78*F_{M5a}+0.61*F_{M5b}.
\end{align*}

Total energy consumption for the AMD server (or the Intel server)
under AR(1) equals the summation of above five consumption contributors
\cite{Lewis2008}.

The models for AMD and Intel severs reveal the issues of adopting linear
regression to obtain individual component contribution.  Consider the
Intel processor as an example.  The coefficients for temperature sensors
are significantly larger than those for the workload-related PeCs, with
those coefficients apparently overbalancing the remaining model
components.  This fact is quite non-intuitive, as one would like to
derive certain physical interpretation on each constant to understand
the behavior of its associated model component.  In addition, other
processor models have negative coefficients for similar model
components, making linear regression deemed unsuitable for such modeling
~\cite{Bertran2010,McCullough2011}.
\begin{comment}
One concern here is that may also be true for the components of the
chaotic times series.  It may be best to point out how we're reasoning
about the totality rather than the components of the system.
\end{comment}

The MARS predictor used in our evaluation was created using the
consolidated data set employed to establish the CAP and the AR(1)
predictors.  It was generated by means of the ARESlab toolkit
\cite{Jekabsons2010}, and the resulting set of splines served as a
predictive tool, as described in Section~\ref{sec:evaluation}.  Note
that ARESlab is a MATLAB toolkit for building piecewise-linear and
piecewise-cubic MARS regression models.  The toolkit was adopted to
build a piece-wise cubic model using the same consolidated training set
employed for creating the AR and the CAP models.  The toolkit output is
a structure which defines the basis functions and associated
coefficients of \equationname~(\ref{eq:mars}) given in
Section~\ref{sec:priorwork} to approximate system dynamics.

The EWMA predictor used in our evaluation was also created using the
consolidated data set employed to established CAP, AR(1), and MARS
predictors.  The predictor was generated using an Exponentially Weighted
Moving Average using the recurrence relation of
\begin{align}
  \label{eq:1}
  S_{1}&=Y_{1}\nonumber\\
  S_{t}&=\alpha Y_{t-1}+(1-\alpha))S_{t-1}, t>1\nonumber
\end{align}
where $Y_{t}$ is an observation of the power consumed at time $t$,
$S_{t}$ is the value of the weighted average at time $t$, and $\alpha$
is a coefficient representing the weighting decrease degree, for
$0\leq \alpha \leq1$.
% Following comment block used by GNU-EMACS and AUCTEX packages
% Please do not remove.
%%% Local Variables: 
%%% mode: latex
%%% TeX-master: "dissertation.tex"
%%% TeX-PDF-mode: t
%%% TeX-source-correlate-mode: t
%%% End: 
