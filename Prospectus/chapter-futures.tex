\chapter{Current State, Plan for Completion, and  Future Plans}
\label{chp:statusandplans}

We presented in \citeN{Lewis2008,Lewis2010,Lewis2011} a full-system
power and thermal mode that provides a run-time system-wide prediction
of energy consumption on server blades as a continuous system of
differential equations.  Using chaotic time series, we construct
discrete approximations of the solution of this system so as to provide
a full system approximation to the thermal behavior of the processor.
In this prospectus, we propose to use this predictor to implement a
prototype of the scheduler described in Chapter~\ref{chp:schedule} that
is predicted to have a measurable reduction in energy consumption and
thermal stress as compared to existing thread schedulers.

\begin{table}
\footnotesize
\caption{Completion Task List}
\label{tab:completeplan}
\centering
% BEGIN RECEIVE ORGTBL completeplan
\begin{tabular}{rl}
Id & Task \\
\hline
\hline
1 & Respond to review comments for \cite{Lewis2011}. \\
2 & Implement scheduler prototype in FreeBSD. \\
3 & Evaluate scheduler performance using parallel benchmarks. \\
4 & Document results and submit to archival journal. \\
5 & Create dissertation from this document \& output from Task 4. \\
6 & Defend dissertation. \\
7 & Respond to comments from committee and Graduate School editor. \\
8 & Submit final version of document. \\
\hline
\end{tabular}
% END RECEIVE ORGTBL completeplan
\end{table}
\begin{comment}
#+ORGTBL: SEND completeplan orgtbl-to-latex :splice nil :skip 0
| Id | Task                                                           |
|----+----------------------------------------------------------------|
|----+----------------------------------------------------------------|
|  1 | Respond to review comments for \cite{Lewis2011}.               |
|  2 | Implement scheduler prototype in FreeBSD.                      |
|  3 | Evaluate scheduler performance using parallel benchmarks.      |
|  4 | Document results and submit to archival journal.               |
|  5 | Create dissertation from this document \& output from Task 4.  |
|  6 | Defend dissertation.                                           |
|  7 | Respond to comments from committee and Graduate School editor. |
|  8 | Submit final version of document.                              |
|----+----------------------------------------------------------------|
\end{comment}

\section{Plan for Completion}
\label{sec:completionplans}
Table~\ref{tab:completeplan} lists the tasks required to complete this
work.  Task~\#1 addresses any required changes that arise from the peer
review of \cite{Lewis2011}.   

A prototype for the scheduler described in Chapter~\ref{chp:schedule}
will be implemented for the FreeBSD operating system
\cite{McKusick2004}.  We use FreeBSD for two reasons: (1) the scheduler
architecture in FreeBSD is well documented and can be simply modified
compared to other choices (such as Linux or OpenSolaris), and (2) the
infrastructure for managing PMCs in FreeBSD has many of the advantages
of OpenSolaris but is not tied to a specific kernel version as is the
Linux infrastructure.

The performance of our prototype is considered in Task~\#3.  The
baseline performance of the scheduler will be determined by examining
the system performance, thermal behavior, and energy consumption of
three standard benchmarks suites: (1) SPEC CPU2006 \cite{spec2006}, (2)
High Performance Linpack(HPL) \cite{Linpack1991}, and (3) the Stream
memory benchmark \cite{McCalpin1995}.  

\section{Future Directions}
\label{sec:future-directions}
The topics of power and thermal management in large scale computing have
only recently begun to receive systematic attention.  As such, we
anticipate that the work in this Prospectus will serve as a starting
point for a longer-term research program.   Examples of longer-term
questions that will need to be addressed include:
\begin{itemize}
\item Our Thermal-Aware Scheduler considers scheduling only within a single
  server blade. High-performance computing applications distribute
  workload across multiple environments using interfaces such as MPI and
  OpenMP.  A topic for further research is how to extend the TCAP and
  Thermal-Aware Scheduling into such environments.
\item A related topic for the future involves the effect of
  operating-system virtualization on Thermal-Aware Scheduling.  Some
  attention has been applied to this area in recent work
  \cite{Merkel2010} but questions remain open considering issues of
  scheduler interference between host and virtual operating systems and
  the impact of virtualization on power management in the clustered,
  grid, and cloud environment.
\item An interesting extension of the system model described in
  \ref{chp:modelstruct} involves considering whether we can quantify the
  energy expended per bit transfer as data moves through the
  system. From this, we can build a more detailed analytical model of
  the thermodynamics of the system than the current practice of
  RC-thermal modeling \cite{Skadron2004}.  
\end{itemize}

\section{Conclusions}
\label{sec:conclusions}
A fast and accurate model for energy consumption and thermal envelope in
a server is critical to understanding and solving the power management
challenges unique in dense servers.  In this paper, we have introduced a
comprehensive model of energy consumption by servers as a continuous
system of differential equations.  The model measures energy input to
the system as a function of the work done for completing tasks being
gauged and the residual thermal energy given off by the system as a
result.  Traffic on the system bus, misses in the L2 cache, CPU
temperatures, and ambient temperatures are combined together to create a
model, which can be employed to manage the processor thermal envelope.

The model serves as a predictive tool by approximating observed
performance metrics in a discrete time series for estimating future
metrics, and thus corresponding energy consumption amounts.  It was
found through experimental validation that commonly used techniques of
regressive time series forecasting, while attractive because of their
simplicity, inadequately capture the non-linear and chaotic dynamics of
metric readings for typical server systems.  Therefore, a chaotic time
series approximation for run-time power consumption is adopted to arrive
at Chaotic Attractor Prediction (CAP), which exhibits polynomial time
complexity.  

We extend the CAP concept to the thermal domain through the concept of
Thermal Chaotic Attractor Prediction (TCAP). Our system model introduced
three metrics that model the system thermal behavior: Thermal Equivalent
of Application, Thermal Efficiency, and Thermal Cost.  TCAP is used to
approximate future values of each of these quantities given the energy
consumption predicted by CAP.  The thermal predictions are used as
optimization metrics in our Thermal-Aware Scheduler for thread
scheduling and load balancing purposes.  Our proposed model is the first
step towards building solutions for power and thermal management in data
centers usually housing many servers.

% Following comment block used by GNU-EMACS and AUCTEX packages
% Please do not remove.
%%% Local Variables: 
%%% mode: latex
%%% TeX-master: "prospectus.tex"
%%% End: 
