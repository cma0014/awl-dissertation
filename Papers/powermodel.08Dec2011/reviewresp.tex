%  $Author: awl8049 $$ 
%  $Date: 2011/11/16 08:28:52 $
%  $Revision: 3.1 $
% 
\documentclass[10pt]{letter} % Uses 10pt
%Use \documentstyle[newcent]{letter} for New Century Schoolbook postscript font
% the following commands control the margins:
\usepackage[paper=letterpaper,
            marginparwidth=0in,
            marginparsep=0.5in,
            margin=1in,
            includemp]{geometry}

% Add the code to give us a biblography environment
\usepackage{enhletter}
% Comment environment support
\usepackage{comment}

% Use fancyhdr and lastpage to get decent page headers and footers
\usepackage{lastpage}
\usepackage{fancyhdr}
\pagestyle{fancy}
\fancyhf{}
\lhead{}
\chead{}
\rhead{}
\lfoot{TACO-2010-27: Review Response: Reviewer \#\thesection, Page \thepage}
\cfoot{}
\rfoot{}
\renewcommand{\headrulewidth}{0pt}
\renewcommand{\footrulewidth}{2pt}

\newenvironment{rviewcomment}
{~\\%
\begin{bfseries}}
{\end{bfseries}}

\newcommand{\rviewresponse}{\textbf{\textit{Response}}:}
\newcommand{\rviewresponses}{\textbf{\textit{Responses}}:}

% Packages for managing per section biblographies
\usepackage[sectionbib]{chapterbib}

\begin{document}

\signature{Adam Wade Lewis}           % name for signature 
\longindentation=0pt                       % needed to get closing flush left
\let\raggedleft\raggedright                % needed to get date flush left
 
 
\begin{letter}{Professor Tom Conte \\
Editor-in-Chief \\
ACM Transactions on Architecture Code Optimization }


\begin{flushright}
{\hfill \large\bf The University of Louisiana at Lafayette}
\end{flushright}
\medskip\hrule height 1pt
\begin{flushright}
\hfill Center for Advanced Computer Studies\\
\hfill P.O. Box 44330 \\
\hfill Lafayette, Louisiana 70504-4330\\
\hfill (337) 482-6338 \\
\hfill \email{awlewis@cacs.louisiana.edu}
\end{flushright} 
%\vfill % forces letterhead to top of page

 
\opening{Dear Professor Conte:} 
 
\noindent We thank the reviewers for the time and effort required for
their thorough reviews of our submission and greatly appreciate
their candid and
valuable comments, which clearly strengthen our manuscript.  We have
taken all comments into full consideration, addressing them in the
revised manuscript, as detailed in sequence.  Please find separate
response documents, one for each of the reviewers, in the enclosed
attachments to this cover letter.
 
\closing{Respectfully Yours,} 
\encl{(1) Response to Reviewer \#1, (2) Response to Reviewer \#2, and
  (3) Response to Reviewer \#3} 
\clearpage
\section{TACO-2010-27: Review Response: Reviewer \#1}
\label{sec-1}
First of all, we greatly appreciate your time and effort put toward
your candid and informative review, which helps to improve the quality
of our submission.  The following lists our responses to your concerns
and comments made on our initial manuscript.  Hopefully, we have addressed
all your concerns and comments to your satisfaction.

\begin{rviewcomment}
  First Comment/Concern:
\end{rviewcomment}
\begin{quote}
\begin{itshape} 
  First, the authors have to address how their work is related to the
  papers I have listed. Since power modeling of servers is a
  well-trodden area, the authors have to highlight how their
  contribution advances the state of the art/knowledge.

  Second, related to the above point, since the per component models are
  not new, the novelty has to be in the way the power modeling itself is
  done and the additional benefits and insights such a modeling approach
  will provide. While the idea of a Chaotic Attractor Predictor appears
  to be novel, none of the quantitative results are convincing that such
  a predictor is an improvement over the prior approaches. The authors
  either need to quantitatively compare, or at least provide a strong
  qualitative discussion about how the CAP predictor advances the
  state-of-the-knowledge on power modeling.
\end{itshape}
\end{quote}

\rviewresponses

We agree with your point, recognizing that there is a large body of
prior work pertinent to this submission.  However, this area is yet to
be well understood, despite being well-trodden.  While the use of linear
auto-regression for constructing power models is widespread in earlier
work, it is demonstrated in our work and also supported by recent
publications~\cite{Kansal2010,Tsirogiannis2010,McCullough2011,Hsu2011}
that significant issues exist with linear regression for power modeling.
It is a major goal of our work to take a different approach aiming to
arrive at better power modeling by considering the problem from a
dynamic standpoint to better capture the chaotic non-linear dynamics
introduced by hardware and OS measures.  To address your concerns and
comments on this point, we have made changes to Section 1 (Paragraphs 1
and 3) and Section 2 (Paragraphs 4, 5, and 6) for clarifying our position
in this regard.

\begin{rviewcomment}
  Second Comment/Concern:
\end{rviewcomment}
\begin{quote}
  \begin{itshape}
    There is a large body of literature on modeling power consumption
    (especially servers) that the authors have overlooked.

    There has been a lot of work on modeling the power and temperature
    of the various components considered in this paper [Skadron et al.,
    ISCA'03], [Gurumurthi et al., ISCA'05], [Liu et al.,
    DAC'08]. [Mesa-Martinez, ASPLOS'10]. The models used by the authors
    are much more simplistic those in these prior work.

    The authors' claim that there is no work that has looked at the
    power profiles of NUMA based architectures is incorrect. See [Ware
    et al., HPCA'10].

    There has also been prior work on using statistical methods to model
    power consumption [Powell et al., HPCA'09].

    Finally, there has been prior work on modeling the power and
    temperature of servers [Heath et al., ASPLOS'06].
  \end{itshape}
\end{quote}

\rviewresponses

Thank you for pointing out those relevant publications,
which are mostly included in Section 2 with appropriate discussion.  Specifically, they are handled as follows.
\begin{description}
\item[\cite{Skadron2003,Skadron2004}] Agreed.  This reference and its
  discussion have been added to Section 2 of this revision.
\item[\cite{Liu2008}] Agreed. This reference has been discussed in
  Section 2.  It should be noted that our work uses the same power model
  as this reference, with both utilizing the DRAM power model found in
  \cite{Micron2007}.
\item[\cite{Gurumurthi2005}] Agreed. This reference plus associated
  discussion has been added to Section 2 and Section 3.3 (Paragraph 1).
\item[\cite{Ware2010}] Agreed. This reference plus associated discussion
  has been added to Section 2.  Like \cite{Ware2010}, our manuscript
  also considered \cite{Brochard2010} and \cite{Rajamani2010} as
  relevant prior work.
\item[\cite{MesaMartinez2010}] Agreed and noted in Section 2 that \cite{Mesamartinez2007} was equally relevant to our work as well.
\item[\cite{Powell2009}] While similar to our manuscript coverage, this
  reference resorted to linear regression and evidenced much of
  ill-behavior reported in our manuscript, as noted in Section 2 of this
  revision.
\item[\cite{Heath2006}] This work has been included and discussed in
  Section 2 of the revised manuscript.
\end{description}


\clearpage
\section{TACO-2010-27: Review Response: Reviewer \#2}
\label{sec-2}
First of all, we greatly appreciate your time and effort put toward
your candid and informative review, which helps to improve the quality
of our submission.  The following lists our responses to your concerns
and comments made on our initial manuscript.  Hopefully, we have addressed
all your earlier concerns and comments raised earlier to your
satisfaction.

\begin{rviewcomment}
  First Comment/Concern:
\end{rviewcomment}
\begin{quote}
\begin{itshape}
I'd like the authors to discuss the time interval used in the evaluation
(for the collection of the PeC and measurement of power). This is also
needed in the appendix which has equations for energy, without
discussing the time interval to which the equations apply. The time
interval determines the possible applications (power capping) for using
the power estimation. For example, actuators that use the parameters
must work on the same time scale to be effective.
\end{itshape}
\end{quote}

\rviewresponses

Agreed. A time interval of 5 seconds was used for our evaluation. 
This value is driven by the sampling accuracy of the
system utilities employed to collect the data.
Discussion about concerning this fact was included in Section 4.2.2
(Paragraph 3, Page 16) and in the Appendix (Paragraph 1, Page 23).

\begin{rviewcomment}
  Second Comment/Concern:
\end{rviewcomment}
\begin{quote}
\begin{itshape}
I could not follow the discussion on calibrating the CAP. The text
claims that p=100 and 4 workloads were run to calibrate the model. I
take this to mean that the 4 workloads together were divided into 100
time intervals (of many seconds or minutes?) and the PeC values and
measured power were averaged over each interval to calibrate the
model. Is this correct? It would be helpful to add a couple of
sentences to be clear how you obtain the 100 vectors required for
calibration.
\end{itshape}
\end{quote}

\rviewresponse

Agreed.  More discussion was added to Section 4.2.2 (Paragraph 2, Page 16)
for clarifying this process.

\begin{rviewcomment}
  Third Comment/Concern:
\end{rviewcomment}
\begin{quote}
\begin{itshape}
The definition of $E_{em}$ is very strange. $E_{em}$ is defined as a component
of server power, but the equation seems to contain the entire servers
power in the form of $V(t) * I(t)$, which contains CPU power (and other
components) which have been already accounted for. In fact, the fan
power itself should be part of $V(t) * I(t)$, the DC power, so it is not
clear why they are added together in the equation for $P_{elect}$.
\end{itshape}
\end{quote}

\rviewresponse

Agreed. Section 3.5 (Page 10) has been adjusted to clarify this issue.

\begin{rviewcomment}
  Fourth Comment/Concern:
\end{rviewcomment}
\begin{quote}
  \begin{itshape}
    Please define SMPS.
  \end{itshape}
\end{quote}


\rviewresponse
 
SMPS is an acronym for \textit{Switched Mode Power Supply}. 


\begin{rviewcomment}
  Fifth Comment/Concern:
\end{rviewcomment}
\begin{quote}
\begin{itshape} 
  I was confused by Section 3.6 which claims that power is controlled,
  but the rest of the paper does not come back to how power is
  controlled or how controlling power is relevant for developing the
  model.
\end{itshape}
\end{quote}

\rviewresponse

Agreed. Section 3.6 has been deleted.

\begin{rviewcomment}
  Sixth Comment/Concern:
\end{rviewcomment}
\begin{quote}
\begin{itshape} 
I'd like a longer discussion of the step-wise process in section 5 for
predicting the distant future. I thought that the function $\hat{f}$ was for
estimating the power of the system for interval $k$, based on PeC
measurements made during time interval $k$. How does one use this to get
power estimates for which PeC has not yet been measured? Are the values
for PeC also being predicted?
\end{itshape}
\end{quote}

\rviewresponse 

Agreed.  Section 5 (Paragraph 1, Page 18) and 5.2 (Paragraph 2, Page 20)
have been adjusted to clarify this process.

\begin{rviewcomment}
  Seventh Comment/Concern:
\end{rviewcomment}
\begin{quote}
\begin{itshape}
I found the equation for $E_{intel_{proc}}$ (in the appendix) surprising
because the coefficients for the temperature of each core is so
different (more than a factor of 70x). It looks like core 1 does not
contribute much to the power of the processor, which is hard to
believe. I would expect them to be more balanced as in the AMD
processor. If all the workload were scheduled on core 1 (core 0
disabled), would these equations still hold?
\end{itshape}
\end{quote}

\rviewresponses 

An interesting observation, as we suspect that this is
an indication of an architectural difference between the processors and
how Solaris carries out load-balancing.  It seems that the OS design
favors balancing workload in the fashion observed by you.  Furthermore,
it is very difficult to reason what is implied by these equations, given
the difficulty in interpreting the physical meaning of regression
coefficients.  We would point you to \cite{McCullough2011} for a
discussion of this topic. Additional text has been added to the Appendix
of this revision (Paragraph 2, Page 24), summarizing some of this work
and how it is compared with our case.  


\clearpage
\section{TACO-2010-27: Review Response: Reviewer \#3}
\label{sec-3}
First of all, we greatly appreciate your time and effort put toward
your candid and informative review, which helps to improve the quality
of our submission.  The following lists our responses to your concerns
and comments made on our initial manuscript.  Hopefully, we have addressed
all your earlier concerns and comments raised earlier to your satisfaction.

\begin{rviewcomment}
    First Comment/Concern: 
\end{rviewcomment}
\begin{quote}
  \begin{itshape}
    Majority of the paper is devoted to introducing the power model,
    subsystem models, CAP, and a variety of other issues such as
    timeseries forecasting, linearity issues, DC vs. AC power
    distribution and alternative performance counters for improving
    power estimation. After these a few examples of estimated and total
    power is shown for some SPEC CPU benchmarks, which do not help us
    (i) validate any of the assumptions for subsystem models; and (ii)
    build any intuition to how the CAP better understands and predicts
    power behavior.

    I recommend reducing some of the lengthy discussions at the
    beginning of this work and removing some tangential points (i.e., AC
    vs. DC, additional counters). It would be much more useful to use
    the remaining space to expand on the observed results and to provide
    some insights to why the CAP method is a better model for system
    power estimation.
  \end{itshape}
\end{quote}

\rviewresponses 

We have attempted to address the concerns and comments made by all
reviewers in this revision. Per the concerns of another reviewer, this
revision adds a few short paragraphs in Sections 1 and 2 to clarify the
unique contributions of our work, and to include and discuss extra prior
articles.  According to your recommendation, we have trimmed down the
first two sections in our original manuscript, including the removal of
Section 3.6 and its associated references.

\begin{rviewcomment}
  Second Concern/Comment:
\end{rviewcomment}
\begin{quote}
\begin{itshape}
  First, I feel the title of this work is somewhat vague. What does
  "time-series approximation of energy consumption estimation" mean? Do
  you fit an estimated energy consumption to a timeseries to predict
  future power behavior? Are you proposing CAP as a timeseries method
  for estimated power, or are you proposing an instantaneous energy
  estimation method based on performance counters; or both? This is not
  clearly explained in the paper. Second, what is "based on server
  workload"? It seems the evaluations are only for Spec CPU benchmarks.
\end{itshape}
\end{quote}

\rviewresponses

We propose an instantaneous method, approximated by CAP, for estimating
power consumption.  Adjustments have been made to Sections 1 (Paragraphs
2, 3, and 4 on Page 2) and 2 (Paragraphs 1, 2, 5, 6, and 7 on Pages
3--6) to clarify the relationship between the title and the contents.

\begin{rviewcomment}
  Third Concern/Comment:
\end{rviewcomment}
\begin{quote}
  \begin{itshape}
    What do you specifically propose for NUMA that prior work omits? How
    do you evaluate the improvements compared to a non-NUMA model?
  \end{itshape}
\end{quote}

\rviewresponses 

Our model relies on tracking data movement across the system as a means
for estimating energy consumption.  The model component for processors
involves estimating the amount of data transmitted on those paths
between cores. Adjustments have been made to Sections 1 (Paragraph 3,
Page 2) and 3.1 (Paragraph 2, Page 7).

\begin{rviewcomment}
  Fourth Concern/Comment:
\end{rviewcomment}
\begin{quote}
  \begin{itshape}
    I am also not sure whether the complexity of CAP is commonly
    necessary for system power estimation. Several prior system-level
    power estimation methods show very good accuracy with very simple
    models, such as utilization-based estimations. Why do they not work
    well in this scenario?
  \end{itshape}
\end{quote}

\rviewresponses

While CAP is more complex than its earlier counterparts (for example,
\cite{Economou2006}), it gains an advantage of exhibiting a better fit
to the dynamics of the system.  Our model captures both the non-linear
and chaotic aspects of the underlying dynamic system better than others.
We have enhanced Section 1 (throughout Pages 2 and 3) and Section 2
(Paragraphs 5, 6, and 7 on Pages 5 and 6) of this revision for
clarification, highlighting the need of a chaotic model to estimate
power consumption more accurately even over short and varying execution
time intervals.

\begin{rviewcomment}
  Fifth Concern/Comment:
\end{rviewcomment}
\begin{quote}
\begin{itshape}
You highlight that your approach accounts for thermal effects in power
estimation. Does this only pertain to processor power estimation, or for
overall system?
\end{itshape}
\end{quote}

\rviewresponse

It focuses on the overall system.  This revision has clarified this
point in Section 1 (Paragraphs 3 and 4).

\begin{rviewcomment}
  Sixth Concern/Comment:
\end{rviewcomment}
\begin{quote}
  \begin{itshape}
    The introduction mentions that prior work "assumes a linear
    relationship between dependent variables and previous data
    points". Do you mean (i) they predict future system behavior based
    on a linear model; or (ii) they use a linear combination of system
    measurements to predict instantaneous power?  For (i) there are some
    prior studies that use statistical or pattern based models to
    predict future system power/performance behavior, as it is commonly
    known that dynamically-varying workload behavior impacts system
    behavior, which does not necessarily vary in a linear way.
  \end{itshape}
\end{quote}

\rviewresponses

We meant the first case in your above list.  Additional discussion and
representative references have been included to Section 1 (Paragraph 1,
Page 3) and Section 2 (Paragraph 5, Pages 5 -- 6) in an attempt to
clarify this issue.

\begin{rviewcomment}
  Set of Concerns and Comments:
\end{rviewcomment}

\textbf{1. $P_{proc}$}
\label{sec-3-6-1}
\begin{quote}
\begin{itshape}
\begin{enumerate}
\item  The derived processor power model (intel) shows a ~100X difference in
       sensitivity to two thermal sensors. Is there any reason for this
       discrepancy? 
\item The model seems quite oblivious to workload and very dependent on
       temperature. Considering temperature does not vary instantaneously,
       while power does, does this model remain valid for different workload
       intensities? For example an idle period after a long running cpu burn
       can result in a higher estimated power than a long-running idle
       period. Or the order of different workload intensities can impact the
       estimated power.
\end{enumerate}
\end{itshape}
\end{quote}

\rviewresponse

Your observation about the sensitivity of the linear models to
temperature, in fact, highlights a point that we are showing about those
models: failing to take into account the chaotic and non-linear dynamics
of the system, one will see odd and unexpected behavior out of the
models, which over-compensate values in certain time windows aiming
to fit a straight-line approximation to a non-linear curve.  We have
added discussion to Section 2 (Paragraphs 5 and 6 on Page 4) and the
Appendix of this revision (Paragraph 4, Page 24) discussing this point.

\textbf{2. $E_{hdd}$ \& $E_{board}$}
\label{sec-3-6-2}
\begin{quote}
\begin{itshape}
I think the HDD power would also depend random vs. sequential writes and
the involvement of disk cache. For the board, I expected more or less
constant power, do you see any significant variations from different
workloads?
\end{itshape}
\end{quote}
\rviewresponses
\begin{description}
\item[\textbf{$E_{hdd}$}:] There are three approaches to approximating hard
  disk energy consumption: an analytical model as proposed in
  \cite{Gurumurthi2005}, an interrupt-based model as proposed in
  \cite{Bircher2011}, and the operating system metric approach as
  adopted in our work.  Our model represents a reasonable first-order
  approximation to this value.
\item[\textbf{$E_{board}$}:] It is possible to treat this value as a constant
  (like what was done in \cite{Bircher2011}).  Given that the value may
  vary from motherboard to motherboard, it is our position, however, to consider
  $E_{board}$ as one model component, with its value determined through
  experimentation.  This position is now noted appropriately in this
  revision (Section 3.4, Page 10).
\end{description}

\begin{rviewcomment}
  Seventh Concern/Comment:
\end{rviewcomment}
\begin{quote}
\begin{itshape}
As I have mentioned previously, it is also quite hard to gauge the
accuracy of the subsystem models without any validation, and
particularly for some of the components Spec CPU would be limiting as it
is designed not to exercise components below memory.
\end{itshape}
\end{quote}

\rviewresponses

The three important points to be emphasized in this work include:
\begin{enumerate}
\item the necessary requirement for chaotic behavior is that the model
  can be expressed in terms of a dynamic system with differential
  equations,
\item this work focuses on the quality of the time series approximation
  provided by the model, and
\item it seems statistically risky to place too much weight on the
  physical interpretation of auto-regressive model components,
  regardless of whether the model is linear or non-linear (see
  \cite{McCullough2011}).
\end{enumerate}
In terms of utilizing the SPEC benchmarks, it is important to remember
that we are looking at the system energy consumption.  While SPEC CPU
2006, as a performance benchmarking suite, focuses on components above
memory, its execution does exercise the entire system when virtual
memory, benchmark bookkeeping, and other interfaces to the operating
system are considered as well.  In that regard, we have included some
description in Section 5.1 (Paragraph 1, page 16--17)
to outline the criteria followed for selecting our benchmarks.

\begin{rviewcomment}
  Eighth Concern/Comment:
\end{rviewcomment}
\begin{quote}
\begin{itshape}
The observations on the chaotic nature of power behavior are quite
interesting. Are the two conditions with Lyapunov and Hurst parameter
sufficient to consider power behavior chaotic? Can you provide some
intuition behind these? I cannot see how the two requisites (high
sensitivity to initial condition \& dense period orbits) hold necessarily
true for power behavior?
\end{itshape}
\end{quote}

\rviewresponse

Agreed. Discussion has been added to Section 2 (Paragraphs 5, 6, and 7),
stating the necessary conditions for exhibiting chaotic behavior.  Also,
references on power electronics which inspired this work are included in
Section 1 (Paragraph 4, Page 3) (\cite{Hamill1997,Tse2002})

\begin{rviewcomment}
  Ninth Concern/Comment:
\end{rviewcomment}
\begin{quote}
\begin{itshape}
Power measurements: The measured power for both systems vary between
60-70W from idle-active. This seems rather low for both servers, and the
delta seems very low for the entire system including CPUs, memory, and
so on. Could you please provide more details on the system
configurations and what is reported.
\end{itshape}
\end{quote}

\rviewresponse

We have consistently seen this kind of measured power draws on the class
of commodity servers used in our experimentation, and the measured
readings are in line with the data provided by server
manufacturers. Our measurements were confirmed by using multiple power
measurement devices from different vendors.

\begin{rviewcomment}
  Tenth Comment:
\end{rviewcomment}
\begin{quote}
\begin{itshape}
Please consider reviewing the paper for language. Here are a few typos
that caught my eye:
\begin{itemize}
\item valid valid \verb --> validate
\item CAT \verb --> CAP
\item 5.2: number of past observations --> number of future observations
\end{itemize}
\end{itshape}
\end{quote}

\rviewresponses

Thanks.  Those typos have been corrected in this revision.


\clearpage
\bibliographystyle{acmtrans}
\bibliography{../overall}

\end{letter}
\end{document}
% The following comment block is used by the different flavors of EMACS and
% the AUCTEX package to manage multiple documents.  In order for AUCTEX
% to understand you're working with multiple files, you should define
% the TeX-master variable as a file local variable that identifies your
% master document.
%
% Please do not remove.
%%% Local Variables: 
%%% mode: latex
%%% TeX-master: "reviewresp.tex"
%%% TeX-PDF-mode: t
%%% End: 
