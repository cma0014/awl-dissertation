\documentclass[11pt]{amsart}
\usepackage{geometry}                % See geometry.pdf to learn the layout options. There are lots.
\geometry{letterpaper}                   % ... or a4paper or a5paper or ... 
%\geometry{landscape}                % Activate for for rotated page geometry
%\usepackage[parfill]{parskip}    % Activate to begin paragraphs with an empty line rather than an indent
\usepackage{graphicx}
\usepackage{amssymb}
\usepackage{epstopdf}
\DeclareGraphicsRule{.tif}{png}{.png}{`convert #1 `dirname #1`/`basename #1 .tif`.png}

\title{Thermally-aware Load Balancing Using Predictive Energy
  Consumption Models in High-Performance Systems}
%\author{The Author}
%\date{}                                           % Activate to display a given date or no date

\begin{document}
\maketitle
%\section{}
%\subsection{}

  Power management techniques developed for mobile and desktop computers
  are not always appropriate for high-performance computing systems
  because resources on these systems tend be fully utilized.As the
  server workload increases, so does the thermal stress on the
  server processors.  Modern processors crudely manage this problem
  through Dynamic Thermal Management (DTM) where the processor monitors
  the die temperature and applies Dynamic Voltage/Frequency Scaling
  (DVFS) to slow the processor and reducing energy consumption and
  thermal impact.  However, DVFS methods entail a heavy
  cost from both a performance and reliability standpoint.

  This paper introduces policy-based, thermally-aware load balancing
  scheduler that utilizes a full-system thermal and energy model to
  proactively predict and minimize hazardous thermal effects. The
  dataflow-based systemwide model estimates the energy consumption and
  thermal impact of an application during its execution. The knowledge
  of the run-time predictive thermal status of the processor allows the
  scheduler time to migrate threads and applications to cooler,
  underutilized cores with minimal performance impact. The model uses
  key micro-architectural event counters and operating system kernel
  statistics to keep track of workload related activity within the
  processor and associated peripherals.

  The scheduler uses the die temperature estimates to adjust the load
  balancing policies to minimize the number of DTM events.  The
  analytical model and the scheduler performance is demonstrated through
  empirical evaluation to outperform previous approaches. This is shown
  by extending an existing power-aware scheduler that is part of the
  OpenSolaris operating system executed on an AMD Opteron processor.  A
  subset of the SPEC CPU2006 benchmark suite is used to simulate
  application server workloads and evaluate scheduler performance.

\end{document}  