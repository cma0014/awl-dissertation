% Created 2010-10-27 Wed 14:13
\documentclass[11pt]{article}
\usepackage[utf8]{inputenc}
\usepackage[T1]{fontenc}
\usepackage{fixltx2e}
\usepackage{graphicx}
\usepackage{longtable}
\usepackage{float}
\usepackage{wrapfig}
\usepackage{soul}
\usepackage{t1enc}
\usepackage{textcomp}
\usepackage{marvosym}
\usepackage{wasysym}
\usepackage{latexsym}
\usepackage{amssymb}
\usepackage{hyperref}
\tolerance=1000
\providecommand{\alert}[1]{\textbf{#1}}

\title{scheduler3.org}
\author{}
\date{2010-10-27 Wed}

\begin{document}

\maketitle

\setcounter{tocdepth}{3}
\tableofcontents
\vspace*{1cm}
\section{THERMAL-AWARE SCHEDULING}
\label{sec-1}
\subsection{Terminology}
\label{sec-1_1}
\subsubsection{thread}
\label{sec-1_1_1}

A schedulable entity reflecting a single flow of execution.
\subsubsection{task}
\label{sec-1_1_2}

A collection of threads running within one address space.
\subsubsection{timeslice}
\label{sec-1_1_3}

A period of time a scheduler intends to schedule a thread.
\subsubsection{quantum}
\label{sec-1_1_4}

The actual execution time utilized by a thread.
\subsubsection{run-queue}
\label{sec-1_1_5}

A queue containing runnable threads of a processor.
\subsubsection{scheduling policy}
\label{sec-1_1_6}
\subsection{Scheduling policies}
\label{sec-1_2}
\subsubsection{constant time scheduling}
\label{sec-1_2_1}

Priority based scheduling algorithm favoring I/O-bound threads
\begin{itemize}

\item Linux O(1) scheduler\\
\label{sec-1_2_1_1}%
\end{itemize} % ends low level
\subsubsection{linear time scheduling}
\label{sec-1_2_2}
\begin{itemize}

\item Round-robin scheduler\\
\label{sec-1_2_2_1}%
\item Multilevel feedback queue\\
\label{sec-1_2_2_2}%
\end{itemize} % ends low level
\subsubsection{Proportional share policies}
\label{sec-1_2_3}
\begin{itemize}

\item fair scheduling/stride scheduling\\
\label{sec-1_2_3_1}%
Fair scheduling: allocation of available CPU time is equally distributed
amongst system users and groups as opposed to equal distribution among
processes.  Common implementation method is to recursively apply
round-robin scheduling strategy at each level of abstraction.
\begin{itemize}

\item Terminology
\label{sec-1_2_3_1_1}%
\begin{itemize}

\item Stride\\
\label{sec-1_2_3_1_1_1}%
The time interval a thread has to wait between its consecutive
executions.

\item Virtual time\\
\label{sec-1_2_3_1_1_2}%
Defines which thread is scheduled next.  Each thread has own concept of
virtual time which is synchronized with a per run-queue global virtual
time when a thread becomes runnable.  The idea is that virtual times of
all threads and the global virtual time of a run-queue are equal at each
distinct point of time.
\end{itemize} % ends low level

\item Start-time fair queuing\\
\label{sec-1_2_3_1_2}%
\item completely fair scheduler\\
\label{sec-1_2_3_1_3}%
Implementation in the most recent Linux kernel. Complexity is O(log N)
where N is the number of tasks in the runqueue.  Choosing a task can be
done in constant time, but reinserting task requires O(log N) because
runqueue is implemented as red/black tree.

\item brain f*** scheduler
\label{sec-1_2_3_1_4}%
\begin{itemize}

\item Loosely based on idea of earliest eligible virtual deadline first\\
\label{sec-1_2_3_1_4_1}%
\item Signal queue of queued but not running processes\\
\label{sec-1_2_3_1_4_2}%
\item Avoid complexity of per-CPU queues\\
\label{sec-1_2_3_1_4_3}%
\item Optimized for mobile and desktop, light NUMA \& fewer than 16 cores\\
\label{sec-1_2_3_1_4_4}%
Balancing of heuristics is a problem.  Gains of local runqueue locking
lost due to need of having grab multiple locks.
\end{itemize} % ends low level
\end{itemize} % ends low level
\end{itemize} % ends low level
\subsection{Approaches}
\label{sec-1_3}
\subsubsection{real-time and embedded systems}
\label{sec-1_3_1}
\subsubsection{adjusting heuristics}
\label{sec-1_3_2}

Use scheduling heuristics to determine how to adjust how processes are scheduled.
\begin{itemize}

\item Heat and Run\\
\label{sec-1_3_2_1}%
Proposed to distribute work amongst available cores until DTM occurs and
migrate work away from the overheated cores.

\item HybDTM\\
\label{sec-1_3_2_2}%
Combine DTM techniques with a thread migration strategy that reduces the
thread priority of jobs on cores that are running hot.

\item ThreshHot\\
\label{sec-1_3_2_3}%
\cite{Yang2008}\cite{Yang2010} used an on-line temperature estimator to
determine what order threads should be scheduled onto cores. This work
analytically demonstrated that you want give preference to those threads
that contribute the most to the increase in the temperature to complete
them as quickly as possible

\item Task Vectors\\
\label{sec-1_3_2_4}%
In \cite{Merkel2008b}, sort taks in each core's run-queue by memory
intensity so as to schedule memory-bound tasks at slower frequencies.
\end{itemize} % ends low level
\subsubsection{avoiding thermal emergencies}
\label{sec-1_3_3}
\begin{itemize}

\item thread migration\\
\label{sec-1_3_3_1}%
Based on adjusting scheduling heuristics 

\item Adjust load balancing based on thread migration\\
\label{sec-1_3_3_2}%
\cite{Coskun2007} move work to the coolest processor.
\end{itemize} % ends low level
\subsection{Opportunities}
\label{sec-1_4}
\subsubsection{assets}
\label{sec-1_4_1}
\begin{itemize}

\item power model
\label{sec-1_4_1_1}%
\begin{itemize}

\item thermal extension\\
\label{sec-1_4_1_1_1}%
\end{itemize} % ends low level
\end{itemize} % ends low level
\subsubsection{Energy as resource}
\label{sec-1_4_2}
\begin{itemize}

\item Prior work
\label{sec-1_4_2_1}%
\begin{itemize}

\item ECOSystem and currentcy \cite{Zeng2002}\\
\label{sec-1_4_2_1_1}%
Currentcy: an abstraction for the energy a system can spend on various
devices.  One unit of currentcy represents the right to consume a
certain amount of energy within a fixed amount of time.   Schedule
processes either with a static-priority policy where system allocates
currentcy to devices according to a static share while currentcy-centric
scheduling adjusts priorities on the basis of ration of consumed
currentcy to entitled currentcy.

\item Cinder capacitors \cite{Rumble2009}\\
\label{sec-1_4_2_1_2}%
Proposed for mobile devices.  Treat energy as equivalent resource to
disk, memory, etc.   Manage in the same way as physical capacitor.

\item Energy limits and task vectors \cite{Bellosa2003}\cite{Klee2008}\cite{Merkel2008}\\
\label{sec-1_4_2_1_3}%
\end{itemize} % ends low level

\item Thermal equivalence of application\\
\label{sec-1_4_2_2}%
Adaptation of the analytical model from journal paper to thermal
domain.  Provides way of dynamically allocating concept of currentcy to
tasks. Use similar model to what was used in Cinder to manage.
\end{itemize} % ends low level
\subsubsection{Avoidance of DTM events}
\label{sec-1_4_3}
\begin{itemize}

\item The time differential\\
\label{sec-1_4_3_1}%
Nanosecond decisions in the scheduler, minutes at the application.
Application decisions for managing temperature and how does this map
into effects seen at the scheduler level?

\item How this notion of fair scheduling affect these schemes?\\
\label{sec-1_4_3_2}%
CFS and BFS both make serious effort to avoid complicated heuristics.
Avoid the complex structures involved with classic UNIX schedulers?
\end{itemize} % ends low level
\subsection{Proposed contributions}
\label{sec-1_5}
\subsubsection{Thermal model}
\label{sec-1_5_1}

A switching activity based thermal model that extends the power model in prior work.
\subsubsection{Thermal load of application (TheA)}
\label{sec-1_5_2}

A means to dynamically gauge application contribution to the thermal load of the system.
\subsubsection{Energy-Aware Fair Scheduling (EAFS)}
\label{sec-1_5_3}
A energy-aware fair scheduling policy that treats energy as a first
level scheduling constraint 

\label{sec:references}
\bibliographystyle{IEEEtran}
\bibliography{scheduler3.bib}


\end{document}
