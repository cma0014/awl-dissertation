% Created 2012-03-19 Mon 17:43
\documentclass[11pt]{article}
\usepackage[utf8]{inputenc}
\usepackage[T1]{fontenc}
\usepackage{graphicx}
\usepackage{longtable}
\usepackage{float}
\usepackage{wrapfig}
\usepackage{soul}
\usepackage{amssymb}
\usepackage{hyperref}
\usepackage[utf8]{inputenc}
\usepackage[T1]{fontenc}
\usepackage{fixltx2e}
\usepackage{graphicx}
\usepackage{longtable}
\usepackage{float}
\usepackage{wrapfig}
\usepackage{soul}
\usepackage{textcomp}
\usepackage{marvosym}
\usepackage{wasysym}
\usepackage{latexsym}
\usepackage{amssymb}
\usepackage{hyperref}
\tolerance=1000
\usepackage{color}
\usepackage{listings}
\providecommand{\alert}[1]{\textbf{#1}}

\title{specsummary.org}
\author{}
\date{2012-03-19 Mon}
\hypersetup{
  pdfkeywords={},
  pdfsubject={},
  pdfcreator={Emacs Org-mode version 7.8.03}}

\begin{document}

\maketitle

\setcounter{tocdepth}{3}
\tableofcontents
\vspace*{1cm}

\section{What do we mean by thread?}
\label{sec-1}
\subsection{Thread of execution in program}
\label{sec-1-1}

\begin{itemize}
\item Compiler optimization
\item Requires the use of compilers that understand how to do this
     optimization and work with some sort of threading library like
     Intel's Thread Building Blocks (Intel compiler) or MPI.
\end{itemize}
\subsection{Operating system thread}
\label{sec-1-2}

\begin{itemize}
\item Scheduling unit in operating system
\item Used to be called ``Lightweight process'' in Unix
\item Applications have to use some sort of threading directive such as
     POSIX pthreads (as used in PARSEC) to implement parallel algorithms.
\end{itemize}
\section{Are the SPEC CPU2006 benchmarks multi-threaded?}
\label{sec-2}

\begin{description}
\item[CONCLUSION] Individual SPEC CPU2006 benchmarks are single-threaded
                  serial implementations.
\item[IMPLICATION] Prior work reporting ``multi-threaded'' results using
                   these benchmarks are in fact reporting on workloads
                   composed of:
\begin{itemize}
\item multiple copies of the same benchmark running at
      the same time
\item multiple benchmarks ran at the same time (approach taken in our paper)
\end{itemize}
\end{description}
\subsection{Supporting documentation}
\label{sec-2-1}

\begin{itemize}
\item From the CPU2006 web site
    (\href{http://www.spec.org/cpu2006/docs/config.html}{http://www.spec.org/cpu2006/docs/config.html})
    ``Note: the SPEC CPU2006 benchmarks do not require multi-threading,
    and do not contain threading directives; but a compiler is allowed
    to try to automatically decompose threads, and there's nothing in
    the run rules to prohibit you from linking with your preferred
    version of a supported malloc library.''
\item The shell scripts provided by SPEC to execute these benchmarks
    provide what is called a ``rate'' metric.   Again from SPEC CPU2006
    web site (\href{http://www.spec.org/cpu2006/Docs/readme1st.html#Q15}{http://www.spec.org/cpu2006/Docs/readme1st.html\#Q15})
    ``For the SPECrate metrics, multiple copies of the benchmarks are run
    simultaneously.  Typically, the number of copies is the same as the
    number of CPUs on the machine, but this is not a requirement.  For
    example, it would be perfectly acceptable to run 63 copies of the
    benchmarks on a 64-CPU machine (thereby leaving one CPU free to
    handle system overhead).''
\item Note the issues with the rate metric
\begin{enumerate}
\item Multiple copies of the same benchmark
\item Note that SPEC considers any runs generated by parallelizing
       compiler as a ``speed'' run, not a ``rate'' run.
\end{enumerate}
\end{itemize}
\subsection{What is the SPEC CPU2006 configuration?}
\label{sec-2-2}

\begin{itemize}
\item The SPEC benchmarks use the GCC compiler.
\begin{enumerate}
\item It does not any sort of automatic thread
        decomposition. Therefore, any sort of multithreading done in
        this case will require to use of threading directives.
\item The benchmarks are built using the ``-o2'' option which only does
        very simple optimizations such as instruction scheduling.  In
        particular, it does not any form of loop-rolling
\end{enumerate}
\end{itemize}
\subsection{SPEC configuration options for number of CPUs and cores?}
\label{sec-2-3}

\begin{itemize}
\item Use of the ``CPU'' terminology is deceptive, it maps to what we are
     calling ``logical cores'' in our work.
\item These configuration options appear to be used for two purposes
\begin{enumerate}
\item Computing the total number of logical CPUs
\item Reporting the hardware configuration in results.
\end{enumerate}
\end{itemize}
\section{CoreHopping}
\label{sec-3}

\begin{itemize}
\item Authors report similar results to ours on single benchmarks!
    ``Figure 4b shows that the temperature differences are consistent
    over all the experimented benchmarks from CPU2006 suite.  On
    average, the peak core temperature is 4 degC higher on core 1.
    Simultaneous multi-threading causes further increase in the on-core
    temperatures.''
\item NOTE: use of term ``simultaneous multi-threading'' in this quote is
    NOT talking about application multi-threading but processor
    virtualization similar to Hyperthreading in the Intel architecture.
\end{itemize}
\subsection{Citation details}
\label{sec-3-1}

   @inproceedings\{Kursun2008,
        Author = \{Kursun, Eren\},
        Booktitle = \{Proc. of the 2008 IEEE International Conference on Computer Design\},
        Date-Added = \{2011-07-13 16:35:38 -0500\},
        Date-Modified = \{2011-07-13 16:35:38 -0500\},
        Doi = \{10.1109/ICCD.2008.4751874\},
        Isbn = \{978-1-4244-2657-7\},
        Month = oct,
        Pages = \{280--285\},
        Publisher = \{IEEE\},
        Title = \{{Variation-aware thermal characterization and management of multi-core architectures\}},
        Url = \{\href{http://dx.doi.org/10.1109/ICCD.2008.4751874}{http://dx.doi.org/10.1109/ICCD.2008.4751874}\}}
\subsection{Benchmarks used:}
\label{sec-3-2}

\begin{itemize}
\item Daxpy, Gromacs, Hmmer, Libquantum, Mcf, Milc, Namd, Perlbench
\end{itemize}
\subsection{Test environment}
\label{sec-3-3}

\begin{itemize}
\item CPU: 1 Quad-core POWER5 processor
\item OS: Linux 2.4 kernel
\end{itemize}
\subsection{Benchmark mix}
\label{sec-3-4}

\begin{itemize}
\item No extensive detail provided about the exact mix
\item Statement: ``Figure 11 shows an example case illustrating the
     reduction in maximum chip temperature through variation-aware
     scheduiling of hot threads from CPU2006 (for example, namd) and
     cold threads (for example, hmmer)''
\item Statement: ``The right bar (again referencing Figure 11 in paper)
     shows the proposed scheme can reduce peak temperatures by 4.5
     degC''.
\begin{itemize}
\item This was reported on peak temperatures of 60-74 degC on testbed
       processor.
\end{itemize}
\end{itemize}
\section{Predict-and-act :: Temperature aware dynamic workload scheduling in multisocket CPU servers}
\label{sec-4}

\begin{description}
\item[NOTE] authors recently published a journal paper with 
            results on a simulated version of the Intel Xeon.
\begin{itemize}
\item These results are more in line with what was observed on
              our testbed.
\end{itemize}
\end{description}
\subsection{Citation details}
\label{sec-4-1}

   @article\{Ayoub2011,
        Author = \{Ayoub, R. and Indukuri, K. and Rosing, T.S.\},
        Doi = \{10.1109/TCAD.2011.2153852\},
        Issn = \{0278-0070\},
        Journal = \{{IEEE Transactions on Computer-Aided Design of Integrated Circuits and Systems\}},
        Keywords = \{control theoretic approach;cooling costs;dynamic thermal management;dynamic workload scheduling;energy savings;fan subsystems;multisocket CPU servers;multitier approach;temperature frequency spectrum;thermal distribution;dynamic scheduling;fans;microprocessor chips;power aware computing;power consumption;processor scheduling;resource allocation;temperature distribution;\},
        Month = \{Sept.\},
        Number = 9,
        Pages = \{1359 -1372\},
        Title = \{{Temperature aware dynamic workload scheduling in multisocket CPU servers\}},
        Volume = 30,
        Year = 2011,
        Bdsk-Url-1 = \{\href{http://dx.doi.org/10.1109/TCAD.2011.2153852}{http://dx.doi.org/10.1109/TCAD.2011.2153852}\}}
\subsection{Test environment}
\label{sec-4-2}

\begin{itemize}
\item CPU: 2 Quad Core Xeon E4440 processor (2 chips, 8 cores)
\item Simulation environment
     HotSpot simulator using Xeon floorplan
\end{itemize}
\subsection{Workloads:}
\label{sec-4-3}
\subsubsection{Starting point}
\label{sec-4-3-1}

\begin{itemize}
\item Conference paper (quoted in current draft) used SPEC CPU2006,
      journal paper use SPEC CPU2000.
\item Workload combinations used in their test
\begin{itemize}
\item Table notation: <number of copies><benchmark
\item For example, 3eon says 3 copies of eon benchmark
\end{itemize}
\end{itemize}

\begin{center}
\begin{tabular}{lll}
 Workload  &  CPU Socket A    &  CPU Socket B       \\
 W1        &  3eon            &  eon + mcf + gcc    \\
 W2        &  2eon + mcf      &  eon + bzip2 + mcf  \\
 W3        &  2bzip2 + 2mcf   &  2bzip2 + 2mcf      \\
 W4        &  2perl + 2eon    &  2gcc + 2mcf        \\
 W5        &  2perl + 2bzip2  &  2gcc + 2mcf        \\
 W6        &  2perl + 2bzip2  &  2gcc + 2mcf        \\
 W7        &  2perl + bzip2   &  gcc+2mcf           \\
 W8        &  perl + 3gcc     &  perl + 2gcc        \\
 W9        &  perl + 3gcc     &  perl + 2gcc        \\
 W10       &  perl + 3gcc     &  perl + 2gcc        \\
 W11       &  perl + 3gzip    &  perl + gcc + gzip  \\
 W12       &  3eon            &  3eon               \\
 W13       &  2eon+mcf        &  2eon+mcf           \\
 W14       &  2bzip2          &  2bzip2             \\
\end{tabular}
\end{center}
\subsubsection{Results}
\label{sec-4-3-2}

\begin{itemize}
\item 3-4 degC savings on temperatures ranging from 36 degC to 42 degC
\item Single digit saving in CPU energy savings, in many cases
      significantly less than 5\% per figure where data is reported
\item Note that work focused on reducing energy consumption due to
      cooling costs (particularly fans) so ways results are presented in
      this paper can be very deceptive as to their actual results.
\end{itemize}


    

\end{document}
