
%  $Description$ 
%
%  $Author: awl8049 $
%  $Date: 2007/07/30 19:20:06 $
%  $Revision: 1.1 $

\documentclass[times, 10pt,onecolumn]{article} 
\usepackage{latex8}
\usepackage{times}
\usepackage{graphicx}
\usepackage{authblk}
%\documentstyle[times,art10,twocolumn,latex8]{article}
\pagestyle{empty}
\begin{document}
\title{Dissteration Working Document}
\author[*]{Adam Lewis}
\affil[*]{Center for Advanced Computer Studies\\ 
University of Louisiana at Lafayette\\
Lafayette, LA 70504, USA\\ 
awlewis@cacs.louisiana.edu
}
\maketitle
\thispagestyle{empty}
\begin{abstract}

\end{abstract}
\section{Problem Statement}
\label{Problem Statement}
Large data centers host hundreds or thousands of servers that require complex
and costly cooling solutions.  Providing proper cooling has become more
difficult as increased component density has resulted in higher energy
consumption and greater heat output.   This issue is compounded in server
clusters as greater numbers of more powerful nodes results in greater energy
consumption across the entire cluster.

Application servers have characteristics that make energy conservation
particularly difficult \cite{Bianchini2004}.  Typical workloads on such
servers lead to intensive CPU and memory use which means energy conservation
mechanisms must not cause execessive overhead under moderate and heavy loads.
In addition, such workloads tend to require applications maintain some form of
soft state that is typically not replicated across nodes in a server
cluster. As result, mechanisms that look to shut down portions of a cluster
would require applications to consider migration of that state across nodes.

On average, the equipment in a single rack enclosure in a data center consumes
about 1.7 kW but the maximum power that can be consumed by a rack enclosure is
over 20 kW. \cite{APC2007a} Consider the case where a typical blade server
consisting of six 7U requires 3 kW of power which means that the entire
enclosure requires 18 kW of power and 18 kW of cooling be provided to the
enclosure. From a cooling standpoint, the data center has to supply 2500 cubic
feet per minute (cfm) of cool air to the enclosure and remove that the same
amount of air from the unit.  Supporting this level of cooling requires
providing almost 8 times of much airflow to the enclosure than what is
typically provided.

Suppose each chassis requires 3 kW of power which means that the entire
enclosure requires 18 kW of power and 18 kW of cooling be provided to the
enclosure. From a cooling standpoint, the data center has to supply 2500 cubic
feet per minute (cfm) of cool air to the enclosure and remove that the same
amount of air from the unit.  Supporting this level of cooling requires
providing almost 8 times of much airflow to the enclosure than what is
typically provided.

The emergence of \it(virtualization), the ability to run applications on
virtual servers where one or more virtual server is mapped to a physical
machine, provides the means to address workload dynamics through the
application of \it(migration).   Migrating a virtual server gives a method to
remap physical resources to virtual servers depending the workload and server characteristics.
% \begin{verbatim}
% From Dr. Tzeng's e-mail ->
% >> You should think about if it make sense to use
% >> virtualization for power reduction via migrating
% >> an execution or a continuous service to lower
% >> power consuming nodes; how and when to do it?
% >> what issues are involved, etc.
% >> An analytic model may be establish for this study as well.
% \end{verbatim}

What opportunities does applying virtualization in the cluster environment
provide to address the problems of energy conservation and thermal management?

\begin{itemize}
\item How: Combine thermal management, virtual machines, and migration to
  reduce temperature and energy consumption?
\item What: Migration, fail-over
\item When: anticipate rather than react to problems
\end{itemize}

\subsection{How}
\label{sec:how}
Prior work in this area has focused on balancing thermal load across a cluster
and then shutting down the overheated nodes. \it{cite required}
Virtualization has been considered as a mean to simplify the migration of
application state by considering the migration of virtual machines from node
to node. \it{cite required}

In this work, we consider the following issues:
\begin{itemize}
\item Is migration required to switch between virtual machines between nodes
  in the cluster?
\item Can virtual machines be used as means of switching between nodes without
  significant impact on performance service level agreements?
\item How do you manage the process? Is it possible to use information in the
  hypervisor to avoid overhead?
\item 
\end{itemize}

\subsection{What}
\label{sec:what}
The workloads being considered in this work have very tight performance
requirements in their service level agreements.  Switching between nodes must
occur in such a way that the service being provided is minimally impacted.

In this work, we consider the following issues:
\begin{itemize}
\item Is migration necessary?  Is it possible to achieve similar effects using
  suspended virtual machines?
\item Is it really necessary to completely shutdown the cluster node to cool
  down the machine?
\end{itemize}

\subsection{When}
\label{sec:Communication}
Much of the work in this area has considered when to migrate a virtual machine
to maximize performance rather than minimize peformance.   Our objective is to
devise a set of algorithms that reduce the overall power consumption without
significant impact to quality of service provided by an application server.

In this work, we consider the following issues:
\begin{itemize}
\item Can we build an analytical model that allows us better anticipate when
  to start moving processes to other machines?
\item What would be the inputs into this model? CPU performance data? sensor
  measurements from individual nodes? sensor measurements from within the
  computer rack.
\end{itemize}

\section{Research Plan}


\section{Related Work}


\section{References}
\nocite{*}
\bibliographystyle{latex8}
\bibliography{ResearchPlan}
\end{document}
\endinput
